\section{Заключение}
В результате выпускной классификационной работы был разработан и успешно внедрен в компании Яндекс распределенный профилировщик Perforator.
Кроме того, была разработана новая эффективная схема организации профилировщика для всех сервисов во внутреннем облаке Яндекса
с использованием новейшей технологии eBPF и подсистемы Linux perf.
Было предложен и успешно реализован механизм раскрутки стека пользовательских приложений в пространстве ядра Linux
с использованием эффективного представления DWARF CFI.

Проект не достиг финальной точки развития.
Можно выделить следующие направления дальнейших внедрений и улучшений:
\begin{itemize}
    \item Полное внедрение новой версии системы.
    \item Реализация чтения TLS.
    \item Поддержка новых языков (Python, Java) и архитектур (ARM64).
    \item Поддержка эффективной агрегации профилей по разным разрезам: по времени, по сервисам, по серверам или моделям CPU.
\end{itemize}
